\section{Conclusion}

In this research, I have developed a comprehensive machine learning framework for predicting traffic accident injury severity in New York City. The theoretical foundation is the ordered logit model, which establishes baseline interpretability before implementing advanced machine learning approaches. This methodology builds on prior validation by Rifat et al. (2024) and Santos et al. (2021), who demonstrated the effectiveness of ensemble methods in crash severity prediction.



Building on literature insights from Aziz et al. (2013) regarding NYC's borough-specific effects and Park et al. (2020) on environmental factors, the analysis of 135,966 collision records reveals that LightGBM with SHAP interpretability provides optimal balance of predictive performance (AUC: 0.675) and actionable insights. Key findings identify rule violations as the strongest fatal crash predictor, complex age-severity relationships, and geographic risk concentration in Queens and Brooklyn, consistent with theoretical expectations from discrete choice modeling.

The framework's practical impact extends to insurance pricing strategies through evidence-based premium adjustments and policy applications supporting targeted resource allocation. The successful integration of behavioral categorization, environmental data, and interpretable machine learning establishes severity prediction as an important complement to traditional frequency-based approaches, offering immediate applications for improving both insurance market efficiency and public safety outcomes while providing a foundation for extension to other urban contexts.