\section{Introduction and Motivation}

New York City has experienced a substantial increase in auto insurance premiums, with year-over-year increases exceeding \$750 in 2024, driven primarily by rising crash severity and associated claim costs. This surge reflects broader national trends where the average cost of full coverage car insurance reached \$2,543 in 2024, marking a 26 percent annual increase that far exceeds the general inflation rate of 2.9 percent. The primary drivers include escalating vehicle repair expenses and critically, worsening accident severity patterns, as evidenced by bodily injury claims increasing 9.2 percent in the same period.

At the heart of this insurance affordability crisis lies the fundamental challenge of injury severity prediction. Unlike accident frequency, the severity of consequences when crashes occur remains poorly understood and inadequately modeled. This gap is particularly problematic because severe injuries and fatalities drive the majority of insurance claim costs, with fatal accidents carrying economic impacts exceeding \$1.8 million per incident according to U.S. Department of Transportation estimates.

From a public safety perspective, New York City's Vision Zero initiative—launched in 2014 with the ambitious goal to eliminate all traffic deaths and serious injuries by 2024—has struggled to achieve its targets despite significant policy efforts and infrastructure investments. The disconnect between policy intentions and outcomes reveals fundamental limitations in current approaches to understanding and predicting crash severity. Traditional safety interventions often rely on historical accident frequency data and basic demographic factors, failing to capture the complex interactions between environmental conditions, driver behaviors, infrastructure characteristics, and temporal patterns that drive severity outcomes.

This study addresses these challenges by developing a comprehensive machine-learning framework for predicting traffic accident injury severity in New York City. I implement a multiclass classification approach that distinguishes between Non-Severe, Severe, and Fatal outcomes. Since these prediction models will be used internally by insurance companies and government agencies for risk assessment and policy planning, the Lucas Critique is weakened because individual drivers remain unaware of the specific algorithmic factors, preventing behavioral changes that could invalidate the model. The research objectives are threefold: to develop an accurate and interpretable machine-learning model capable of predicting traffic accident severity, to identify and quantify the key factors driving severity outcomes, and to translate model insights into practical recommendations for insurance industry pricing strategies and public safety policy implementation.