\section{Model Limitations and Future Research}

\subsection{Current Limitations}

The AUC of 0.675 is indicative of weak predictive capability—only slightly better than chance, as a coin toss would deliver 50 percent accuracy. Thus, improvements are required. Enhanced weather data could help capture local variations that influence crash severity beyond the borough-level weather data currently used. For example, New York City exhibits diverse microclimates and urban heat island effects that create meaningful within-city variation. Incorporating driver history and experience could also provide individual-level behavioral insights, complementing the incident-specific categories currently employed and enabling more precise risk assessment.



Road infrastructure and design features could capture systematic factors that influence severity outcomes beyond the geographic effects currently in the model. Information about road geometry, intersection design, traffic control systems, and infrastructure condition could explain the geographic patterns found in the analysis while providing additional predictive power for severity outcomes.

The analysis uses a three year window designed to avoid COVID-19 disruptions that created significant departures from normal traffic patterns during 2020-2021. While this approach ensures the model reflects normal operating conditions, longer historical periods could provide additional insights into trends and seasonal variations. Additionally, the analysis is limited to reported crashes meeting NYC's reporting requirements and focuses on at-fault drivers, which may exclude information about passenger and pedestrian severity patterns that could inform broader safety strategies.

\subsection{Future Research Directions}

Future research should examine longer historical periods to understand how crash severity patterns evolve over time and evaluate the effectiveness of safety interventions. Incorporating detailed infrastructure variables such as road design, intersection types, and traffic signal configurations could help explain why certain areas like Queens and Brooklyn consistently show higher severity risks. Additionally, tracking individual drivers over multiple years would provide insights into how age, experience, and behavioral changes affect crash outcomes, enabling more personalized risk assessment for insurance applications.


Testing this framework in other cities would help determine which findings are specific to New York versus universal patterns that apply broadly. The discrete choice theory foundation provides a solid basis for adaptation to different urban environments with varying traffic patterns and infrastructure characteristics. Developing practical applications of this research, such as real-time risk assessment tools that incorporate current weather conditions and traffic data, could benefit both insurance companies and transportation agencies in making data-driven decisions about pricing and safety interventions.