\section{Insurance Industry Applications}

The SHAP analysis provides quantitative evidence for implementing differentiated pricing strategies based on empirically validated risk factors. Rule violations demonstrate the strongest predictive power for fatal outcomes, with consistently positive SHAP values across all severity categories, supporting premium loadings for drivers with violation histories. Geographic risk stratification reveals systematic severity differences across NYC boroughs, with Queens and Brooklyn consistently showing elevated risk profiles, providing actuarial justification for location-based rating factors that reflect underlying infrastructure and traffic density differences.

Age-related risk patterns exhibit complexity requiring sophisticated pricing approaches, as the SHAP analysis reveals that teenage drivers show elevated fatal crash risk while drivers over 65 demonstrate increased severe injury risk but reduced fatal risk. This divergent pattern suggests that traditional linear age rating may be suboptimal, supporting segmented pricing strategies that account for distinct risk profiles across demographic categories. Vehicle characteristics present nuanced considerations, as the counterintuitive finding that newer vehicles show protective effects for severe injuries but increased fatal crash exposure suggests that simple vehicle age discounts may inadequately capture risk dynamics.

Environmental variables, particularly temperature and humidity, rank among the top predictive features, suggesting systematic integration of weather data into risk assessment processes for dynamic pricing adjustments and seasonal risk management strategies. The behavioral classification system enables evaluation of driver risk profiles beyond traditional demographic factors, allowing identification of drivers with histories of rule violations and unsafe maneuvers for enhanced risk assessment. SHAP interpretability provides regulatory compliance advantages by offering transparent, quantifiable explanations for pricing decisions, demonstrating empirical relationships between specific risk factors and severity outcomes while supporting actuarial justification requirements and maintaining customer transparency in rating factor applications.