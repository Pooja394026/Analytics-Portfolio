\section{Public Safety and Policy Applications}

The geographic concentration of severe and fatal crashes in Queens and Brooklyn, validated across multiple analytical approaches, provides clear targeting criteria for infrastructure investment and enforcement resource allocation, enabling transportation authorities to prioritize interventions in areas where severity reduction potential is empirically demonstrated to be highest. The dominance of rule violations as fatal crash predictors suggests reorienting enforcement strategies from general traffic monitoring toward compliance-focused interventions, with speed enforcement, traffic signal compliance monitoring, and right-of-way violation prevention emerging as evidence-based priorities that directly address the strongest predictors of fatal outcomes identified in the analysis.

Environmental risk patterns support implementing weather-responsive safety protocols, as the prominence of temperature and humidity as severity predictors indicates that dynamic safety measures—such as enhanced enforcement during high-risk weather conditions or real-time driver advisories—could provide measurable severity reduction benefits. The finding that two-wheel and micro-mobility vehicles show dramatically higher severe injury proportions suggests prioritizing protected infrastructure for vulnerable road users, while the complex relationship between vehicle age and severity outcomes indicates that safety benefits from fleet modernization may be more nuanced than traditionally assumed.

The severity prediction framework enables proactive rather than reactive safety planning by identifying risk patterns before they manifest in increased crash frequencies, allowing identification of emerging risk concentrations based on environmental, behavioral, and demographic factors that precede crash occurrence. Age-specific risk patterns support differentiated safety interventions, with young driver fatal crash prevention programs emphasizing rule compliance and risk awareness, while elderly driver programs focus on severe injury prevention through enhanced infrastructure design and vehicle safety technology promotion. The behavioral classification system provides a foundation for targeted safety campaigns, as understanding that different risky behaviors influence severity through distinct mechanisms enables developing specific interventions for distracted driving, impaired driving, and aggressive driving behaviors based on their empirically demonstrated severity impacts.